\section{Introduction}

The Eiffel language \cite{EcmaEiffel} \cite{Meyer09} has a lot of different persistence libraries, and every solution has its own advantages and drawbacks.
A database such as MySQL \cite {MySQL} for example is quite fast, but it comes at the expense of the object-relational impedance mismatch \cite{ORImpedance}.
A serialization library on the other hand is easier to handle for the programmer, but its performance rapidly decreases for large data sets.

It is very hard to switch from one persistence solution to another, because all have their own interface.
Even changing for example the data\-base from MySQL to Oracle \cite{Oracle} is usually very hard to achieve, if only because their SQL dialects are different.

To overcome such problems we have developed a new library called ABEL, which is the acronym for ``A better Eiffelstore library''.
ABEL tries to unify existing persistence libraries unter a simple and yet powerful API, which is completely transparent to the actual storage mechanism.


\section{Overview}
This thesis is basically splitted into two parts: The API tutorial and the technical documentation.

In the first part you will be introduced to the basic operations of the API, like the CRUD (Create, Read, Update, Delete) operations or transaction handling.

The second part is an introduction to the general architecture of ABEL and some selected topics like the object-relational mapping layer or the main interfaces for backend abstraction.