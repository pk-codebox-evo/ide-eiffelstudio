\documentclass[a4paper,12pt]{report}
\usepackage{url}
\usepackage{enumitem}
\usepackage{palatino}
\usepackage[cmex10]{amsmath}
\usepackage{stmaryrd,amssymb}
\usepackage{graphicx}
\usepackage{amsfonts}
\usepackage{rotating}
\usepackage{listings}
\usepackage{xcolor}
\usepackage{url,boxedminipage,listings}
\usepackage{booktabs}
\usepackage{rotating}
\usepackage{array}
\usepackage{color,url,varioref,xcolor}
\usepackage{alltt,epsfig,comment}
\usepackage{supertabular,fancyhdr}
\usepackage{multirow}
\usepackage{url}
\usepackage{xspace}
\usepackage{dsfont}
\usepackage[english]{babel}
\usepackage{todonotes}

\usepackage{pdfpages}
\usepackage[font={footnotesize,sl}, labelfont=bf] {caption}

\usepackage{tikz}
\usetikzlibrary{arrows}
\usetikzlibrary{automata}
\usetikzlibrary{positioning}
\usetikzlibrary{er}

% HAS TO BE LAST PACKAGE:
\usepackage[pdftex,bookmarks=true,pageanchor=false]{hyperref}



% Eiffel Code Stuff
\lstset{language=OOSC2Eiffel,basicstyle=\ttfamily\small}
\definecolor{codebg}{rgb}{0.95,0.95,0.95}
\setlength{\headheight}{28pt}

% Inline Eiffel Code
\let\e\lstinline
% Eiffel Identifiers or expressions
\newcommand{\eid}[1]{\textsl{{\color[HTML]{000000} #1}}}
% Eiffel Class names
%\newcommand{\ecl}[1]{\textsl{{\color[HTML]{3333FF} #1}}}
\newcommand{\ecl}[1]{\textsl{{\color[HTML]{000000} #1}}}
% Eiffel Keywords
%\newcommand{\ekey}[1]{\textbf{{\color[HTML]{333399} #1}}}
\newcommand{\ekey}[1]{\textbf{{\color[HTML]{000000} #1}}}

\newcommand{\distance}[2]{\ensuremath{#1\leftrightarrow#2}}

\lstset{escapechar=\$}

\newcommand{\blankpage}{
\newpage
\thispagestyle{empty}
\mbox{}
\newpage
}


\title {The ABEL Persistence Library Tutorial}
\author {
	\textbf{Written by:} Roman Schmocker\\
	\textbf{Reviewed by:} Marco Piccioni\\\\
	\textbf{Last updated:}
}


\begin{document}

%\pagenumbering{roman}
%\includepdf {includes/title_page}
\maketitle

%\blankpage

%\section*{Abstract}

ABEL is a persistence library that unifies existing solutions under a simple and transparent API.
The API fully supports the CRUD operations, transactions, and has some advanced features like result filtering based on some criteria.

The library contains a powerful framework with an object-relational mapping layer at the core that allows to adapt ABEL to basically any existing persistence solution.

%\blankpage

\tableofcontents

%\chapter{Introduction}
\pagenumbering{arabic}


%\chapter{Introduction}
%The initial part of the introduction was taken from the abstract. Maybe some modifications are to be done here...
Debugging applications is very tedious work. One of the hardest steps to find a fault in a program is to reproduce the steps that lead to the program failure. Capture and replay aims at making the execution of a program repeatable, removing this burden from the developer. In order to be able to replay an application, it is necessary to isolate the deterministic part of the application that needs to be replayed (observed part) and the non-deterministic parts, like user input, storage or network (unobserved part). Once the behavior of the unobserved part can be captured and replayed, it is also possible to replay the observed part. Although capture and replay is commonly known as capturing mouse and keyboard events, there are approaches that go further and capture events from all interactions with the environment. All these approaches have one thing in common: They have a fixed border between observed and unobserved part.

Selective capture and replay \cite{orso05may} makes it possible to individually define the observed part. Because the information flow is not equally intense between all parts of a program, this offers the possibility to optimize the amount of information that needs to be captured. To reduce the amount of captured data, selective capture and replay uses a special technique: whenever an object passes the border, only its type and a unique identifier is recorded, unless the passed object is from a basic type. It is possible to show, that this suffices to replay the observed part. This technique ensures that the amount of recorded data is linear to the number of variables passed through the border, in contrast to other techniques which capture the whole data that is passed

We have developed a selective capture and replay framework for Eiffel. It allows capture and replay functionality for whole applications with some changes in the runtime and modifications in the libraries that are significantly smaller than for traditional approaches.

The original implementation for selective capture and replay was made for Java. The technique proposed in the original paper can not be completely applied to Eiffel because some language features differ between Eiffel and Java. Some of these differences make an implementation easier, for example it is not possible to change attributes of other classes, and some make it harder, for example multiple inheritance. Selective capture and replay relies on code instrumentation to capture accesses across the border. The original implementation instruments method and attribute accesses on the caller (client) side, whereas we instrument code at the callee side. The instrumentation our modified compilers adds to the program is the same for both the capture and the replay run. Hence no recompilation between the two phases is necessary. Depending on the program, compilation times can last for tens of minutes up to hours. For some capture and replay applications a recompilation can be prohibitive, therefore this feature is necessary. The original implementation instruments the program at the bytecode level and consequently doesn't need a complete recompile when the changing the code instrumentation.

  This thesis provides an implementation that shows the feasibility of selective capture and replay in Eiffel, rather than the possible performance. Nonetheless, some performance measurements will be presented, based on a simple example application.




\chapter{Introducing ABEL}
ABEL (A Better EiffelStore Library) is an object-oriented persistence library written in Eiffel and aiming at seamlessly integrating various kinds of data stores.
 
\section{Setting things up}
We are assuming you have checked out the ABEL code from the EiffelStudio SVN repository\footnote{https://svn.eiffel.com/eiffelstudio/branches/eth/eve/Src/library/abel}, and have EiffelStudio installed. Launch it and in the initial window choose "tutorial\_project". If it is not there just push "Add project" and navigate to the location where you downloaded ABEL, and look for the \emph{tutorial\_project.ecf} project file in \emph{abel/apps/sample/tutorial/}. You can then load and compile the project. To be able to compile the ABEL tutorial you don't need particular dependencies, because we are using an in-memory database simulating a relational database. If you want to experiment with ABEL's support for a full-fledged relational back-end (like MySQL or SQLite, see Chapter~\ref{chapter:advanced_initialization}), you need to install the databases and the appropriate drivers.

\section{Getting started}
We will be using \lstinline!PERSON! objects to show the usage of the API. 
In the source code below you will see that ABEL handles objects "as they are", meaning that to make them persistent you don't need to add any dependencies to their class source code.

\begin{lstlisting}[language=OOSC2Eiffel, captionpos=b, caption={The PERSON class}, label={lst:person_class}]
class PERSON

create
	make

feature {NONE} -- Initialization

	make (first, last: STRING)
			-- Create a newborn person.
		require
			first_exists: not first.is_empty
			last_exists: not last.is_empty
		do
			first_name := first
			last_name := last
			age:= 0
		ensure
			first_name_set: first_name = first
			last_name_set: last_name = last
			default_age: age = 0
		end

feature -- Basic operations

	celebrate_birthday
			-- Increase age by 1.
		do
			age:= age + 1
		ensure
			age_incremented_by_one: age = old age + 1
		end

feature -- Access

	first_name: STRING
		-- The person's first name.

	last_name: STRING
		-- The person's last name.

	age: INTEGER
		-- The person's age.

invariant
	age_non_negative: age >= 0
	first_name_exists: not first_name.is_empty
	last_name_exists: not last_name.is_empty
end

\end{lstlisting}

There are three very important classes in ABEL:
\begin{itemize}
 \item The deferred class \lstinline!PS_REPOSITORY! provides an abstraction to the actual storage mechanism.
 \item The \lstinline!PS_CRUD_EXECUTOR! class is, as the name suggests, responsible to execute CRUD (Create Read Update Delete) commands. Every \lstinline!PS_CRUD_EXECUTOR! object works with a \lstinline!PS_REPOSITORY!.

 \item The \lstinline!PS_OBJECT_QUERY [G]! class is used to describe a read operation over objects of type \lstinline!G!. You can execute such a query in the \lstinline!PS_CRUD_EXECUTOR!. 
	The result will be objects of type \lstinline!G!.

 
\end{itemize}
To start using the library, we first need to create an object of type\\
\lstinline!PS_REPOSITORY!. In this case we will be creating a more specific object of type \lstinline!PS_IN_MEMORY_REPOSITORY!. \\
As a second step, we need to create an object of type  \lstinline{PS_CRUD_EXECUTOR}, useful to execute CRUD operations. To create it, we will pass as an argument to its creation feature the previously created repository.

\begin{lstlisting}[language=OOSC2Eiffel, captionpos=b, caption={The TUTORIAL class}, label={lst:tutorial_class}]
class TUTORIAL

create
	make

feature {NONE} -- Initialization

	make
		-- Set up a simple in-memory repository.
		local
			repository: PS_IN_MEMORY_REPOSITORY
		do
			create repository.make_empty
			create executor.make (repository)
		end

feature
	
	executor: PS_CRUD_EXECUTOR
		-- The CRUD executor used throughout the tutorial.

end
\end{lstlisting}
We will use this class throughout the tutorial. You can assume that the Eiffel features listed in this tutorial are located inside the \lstinline!TUTORIAL! class, if they are not enclosed in another class declaration.\\ 
We encourage you to test the features shown in this tutorial by calling them from feature \lstinline{explore} in class \lstinline!TUTORIAL!.
\chapter{Basic operations}

\section{Inserting}

You insert an object in the repository using feature \lstinline{insert} in class\\ 
\lstinline{PS_CRUD_EXECUTOR}. Let's add three new persons to the database in feature \lstinline{explore}:
\begin{lstlisting}[language=OOSC2Eiffel, captionpos=b, caption={Insertion code.}, label={lst:tutorial_insert}]
	explore
			-- Tutorial code.
		local
			p1, p2, p3: PERSON
		do
			-- Insert 3 new persons in the database
			create p1.make ("Albo", "Bitossi")
			p1.celebrate_birthday
			executor.insert (p1)
			create p2.make ("Berno", "Citrini")
			p2.celebrate_birthday
			p2.celebrate_birthday
			p2.celebrate_birthday
			executor.insert (p2)
			create p3.make ("Dumbo", "Ermini")
			executor.insert (p3)			
		end
\end{lstlisting}

\section{Querying}
\label{section:querying}
A query for objects is done by creating a \lstinline!PS_OBJECT_QUERY [G]! object and executing it using features of \lstinline!PS_CRUD_EXECUTOR!.
The generic parameter \lstinline!G! denotes the type of objects that should be queried.

After a successful execution of the query, you can find the result in the iteration cursor \lstinline{result_cursor} in class \lstinline{PS_OBJECT_QUERY}. The feature \lstinline{simple_query} below shows how to get a list of persons from the repository:
%Having an iteration cursor as a result has several advantages, e.g. support for lazy loading or the across syntax, as you will see in the next example:

\begin{lstlisting}[language=OOSC2Eiffel, captionpos=b, caption={A simple query.}, label={lst:simple_query}]
	simple_query: LINKED_LIST [PERSON]
		-- Query all persons from the current repository.
		local
			query: PS_OBJECT_QUERY [PERSON]
		do
			create Result.make
			create query.make
			executor.execute_query (query)

			across query as	query_result
			loop
				Result.extend (query_result.item)
			end
		end
\end{lstlisting}
We now add in feature \lstinline{explore} the code to print the linked list returned by feature \lstinline{simple_query}:  
\begin{lstlisting}[language=OOSC2Eiffel, captionpos=b, caption={Printing the query result.}, label={lst:tutorial_print_result}]
	explore
			-- Tutorial code.
		local
			p1, p2: PERSON
		do
			-- Same code as before
			-- Query the database and print result
			print_result (simple_query)
		end
\end{lstlisting}
Feature  \lstinline{print_result} takes the linked list result of the query and prints all its elements.
Usually the result of such a query is very big, and you are probably only interested in objects that meet certain criteria, e.g. all persons of age 20. You can read more about it in Chapter ~\ref{sec:advanced_queries}.

Please note that ABEL does not enforce any kind of order on a query result.

%\begin{comment}
%ABEL can also filter the query results in advance so you only get a result set that meets certain criteria: 
%
%\begin{lstlisting}[language=OOSC2Eiffel, captionpos=b, caption={}, label={lst:simple_filtered_query}]
%	simple_filtered_query (name: STRING; age: INTEGER): detachable PERSON
%		-- Query a person object from the current repository
%		local
%			query:PS_OBJECT_QUERY[PERSON]
%			criterion:PS_PREDEFINED_CRITERION
%		do
%			create query.make
%			create criterion.make ("last_name", "=", name)
%			query.set_criterion (criterion)
%
%			from
%				executor.execute_query (query)
%			until 
%				query.result_cursor.after
%			loop
%				if query.result_cursor.item.age = age then 
%					Result:= query.result_cursor.item
%				end
%			end
%		end
%\end{lstlisting}
%
%This is just a very simple example for a query with a certain criterion.
%ABEL has a powerful mechanism that also supports a logical combinations of multiple criteria, or using agents for filtering.
%You can read more about criteria in section XY.
%
%\end{comment}

\section{Updating}

Updating an object is done through feature \lstinline{update} in \lstinline{PS_CRUD_EXECUTOR}. Let's update the \lstinline{age} attribute of Berno Citrini by celebrating his birthday:

\begin{lstlisting}[language=OOSC2Eiffel, captionpos=b, caption={Printing the query result.}, label={lst:tutorial_print_result}]
	explore
			-- Tutorial code.
		local
			p1, p2: PERSON
		do
			-- Same code as before
			-- Update an existing person in the database and print the result again
			p2.celebrate_birthday
			executor.update (p2)
			print_result (simple_query)		
		end
\end{lstlisting}
The object to update needs to be previously known to ABEL through an insert or a successful query (see Section~\ref{section:dealing_with_known_objects}).

\section{Deleting}
\label{section:simple_delete}

Deletion is done through feature \lstinline{delete} in \lstinline{PS_CRUD_EXECUTOR}.
Let's now delete Albo Bitossi from the database:
\begin{lstlisting}[language=OOSC2Eiffel, captionpos=b, caption={Deleting an object.}, label={lst:tutorial_print_result}]
	explore
			-- Tutorial code.
		local
			p1, p2: PERSON
		do
			-- Same code as before
			-- Delete Dumbo Ermini from the database and print the result again
			executor.delete (p3)
			print_result (simple_query)
		end
\end{lstlisting}
Another way to delete objects is described in Section ~\ref{section:deletion_query}.
\section{Dealing with Known Objects}
\label{section:dealing_with_known_objects}

ABEL keeps track of objects that have been inserted or queried.
This is important because in case of an update or delete, the library internally needs to map the object in the current execution of the program to its specific entry in the database.

Because of that, you can't update or delete an object that is not yet known to ABEL.
As an example, the following two functions will fail:

\begin{lstlisting}[language=OOSC2Eiffel, captionpos=b, caption={Failing updates and deletes.}, label={lst:failing_update_delete}]
	failing_update
		-- Try and fail to update a new person object
		local
			a_person: PERSON
		do
			create a_person.make ("Bob", "Barath")
			executor.update (a_person)
				-- Results in a precondition violation
		end

	failing_delete
		-- Try and fail to delete a new person object
		local
			a_person:PERSON
		do
			create a_person.make ("Cersei", "Lannis")
			executor.delete (a_person) 
				-- Results in a precondition violation
		end
\end{lstlisting}

Please note that there's another way to delete objects, described in Section ~\ref{section:deletion_query}, which doesn't have this restriction.

The feature \lstinline{is_persistent} in \lstinline!PS_CRUD_EXECUTOR! can tell you if a specific object is known to ABEL and hence has a link to its entry in the database.

\chapter{Advanced Queries}
\label{sec:advanced_queries}

\section{The query mechanism}

As you already know from Section~\ref{section:querying}, queries to a database are done by creating an object of type  \lstinline!PS_OBJECT_QUERY[G]! and using it from within a \lstinline!PS_CRUD_EXECUTOR!.
The actual value of the generic parameter \lstinline!G! determines the type of the objects that will be returned, including any conforming type (e.g. descendants of \lstinline!G!).

ABEL will by default load an object completely, meaning all objects that can be reached by following references will be loaded as well (see also Chapter ~\ref{chapter:references}).

\section{Criteria}

You can filter your query results by setting criteria in the query object, using feature \lstinline{set_criteria} in \lstinline{PS_OBJECT_QUERY}.
There are two types of criteria: predefined and agent criteria.

\subsection{Predefined Criteria}
When using a predefined criterion you pick an attribute name, an operator and a value. 
During a read operation, ABEL checks the attribute value of the freshly retrieved object against the value set in the criterion, and filters away objects that don't satisfy the criterion.

Most of the supported operators are pretty self-describing (see class \lstinline{CRITERION_FACTORY} in Section~\ref{sec:creating_criteria_objects}).
An exception could be the \lstinline!like! operator, which does pattern-matching on strings.
You can provide the \lstinline!like! operator with a pattern as a value. The pattern can contain the wildcard characters \lstinline!*! and \lstinline!?!.
The asterisk stands for any number (including zero) of undefined characters, and the question mark means exactly one undefined character.

You can only use attributes that are strings or numbers, but not every type of attribute supports every other operator. Valid combinations for each type are:

 \begin{itemize}
  \item Strings: =, like
  \item Any numeric value: $=, <, <=, >, >=$
  \item Booleans: =
 \end{itemize}

Note that for performance reasons it is usually better to use predefined criteria, because they can be compiled to SQL and hence the result can be filtered in the database.

\subsection{Agent Criteria}

An agent criterion will filter the objects according to the result of an agent applied to them.

The criterion is initialized with an agent of type \lstinline!PREDICATE [ANY, TUPLE [ANY]]!. 
There should be either an open target or a single open argument, and the type of the objects in the query result should conform to the agent's open operand. For an example see Section~\ref{sec:creating_criteria_objects}.

\subsection{Creating criteria objects}
\label{sec:creating_criteria_objects}
The criteria instances are best created using the \lstinline!CRITERION_FACTORY! class.

The main features of the class are the following: 

\begin{lstlisting}[language=OOSC2Eiffel, captionpos=b, caption={The CRITERION\_FACTORY class interface}, label={lst:factory_interface}]
class
	PS_CRITERION_FACTORY
create
	default_create

feature -- Creating a criterion

	new alias "[]" (tuple: TUPLE [ANY]): PS_CRITERION
		-- Creates a new criterion according to a `tuple'
		-- containing either a single PREDICATE or three 
		-- values of type [STRING, STRING, ANY].

	new_agent (a_predicate: PREDICATE [ANY, TUPLE [ANY]]): PS_CRITERION
		-- Creates an agent criterion.

	new_predefined (object_attribute: STRING; 
		operator: STRING; value: ANY): PS_CRITERION
		-- Creates a predefined criterion.

feature -- Operators

	equals: STRING = "="

	greater: STRING = ">"

	greater_equal: STRING = ">="

	less: STRING = "<"

	less_equal: STRING = "<="

	like_string: STRING = "like"

end
\end{lstlisting}

Assuming you have an object \lstinline{f: PS_CRITERION_FACTORY}, to create a new criterion you have  two possibilities:

 \begin{itemize}
  \item The "traditional" way
  \begin{itemize}
  \item \lstinline!f.new_agent (agent an_agent)!
  \item \lstinline!f.new_predefined (an_attr_name, an_operator, a_val)!
  \end{itemize}
  \item The "syntactic sugared" way
  \begin{itemize}
  \item \lstinline!f[[an_attr_name, an_operator, a_value]]!
  \item \lstinline!f[[agent an_agent]]!
  \end{itemize} 
  \end{itemize}
caption={The CRITERION\_FACTORY interface}
\begin{lstlisting}[language=OOSC2Eiffel, captionpos=b, caption={Different ways of creating criteria.}, label={lst:factory_usage}]

	create_criteria_traditional : PS_CRITERION
		-- Create a new criteria using the traditional approach.
		do
			-- for predefined criteria
			Result:= 
				factory.new_predefined ("age", factory.less, 5)

			-- for agent criteria
			Result := 
				factory.new_agent (agent age_more_than (?, 5))
		end

	create_criteria_double_bracket : PS_CRITERION
		-- Create a new criteria using the double bracket syntax.
		do
			-- for predefined criteria
			Result:= factory[["age", factory.less, 5]]

			-- for agent criteria
			Result := factory[[agent age_more_than (?, 5)]]
		end			

	age_more_than (person: PERSON; age: INTEGER): BOOLEAN
		-- An example agent
		do
			Result:= person.age > age
		end

\end{lstlisting}

\subsection{Combining criteria}

You can combine multiple criterion objects by using the standard Eiffel logical operators. 
For example, if you want to search for a person called ``Albo Bitossi'' with $age <= 20$, you can just create a criterion object for each of the constraints and combine them:  

\begin{lstlisting}[language=OOSC2Eiffel, captionpos=b, caption={Combining criteria.}, label={lst:search_albo_bitossi}]

	composite_search_criterion : PS_CRITERION
		-- Combining criterion objects.
		local
			first_name_criterion: PS_CRITERION
			last_name_criterion: PS_CRITERION
			age_criterion: PS_CRITERION
		do
			first_name_criterion:= 
				factory[[ "first_name", factory.equals, "Albo" ]]

			last_name_criterion := 
				factory[[ "last_name", factory.equals, "Bitossi" ]]

			age_criterion := 
				factory[[ agent age_more_than (?, 20) ]]
			
			Result := first_name_criterion and last_name_criterion and not age_criterion

			-- using double brackets for compactness. 
			Result := factory[[ "first_name", "=", "Albo" ]] 
				and factory[[ "last_name", "=", "Bitossi" ]] 
				and not factory[[ agent age_more_than (?, 20)  ]]
		end
\end{lstlisting}

ABEL supports the three standard logical operators \lstinline!AND!, \lstinline!OR! and \lstinline!NOT!. 
The precedence rules are the same as in Eiffel, which means that \lstinline!NOT! is stronger than \lstinline!AND!, which in turn is stronger than \lstinline!OR!.

We can now add the necessary code to feature \lstinline{explore}:  
\begin{lstlisting}[language=OOSC2Eiffel, captionpos=b, caption={Invoking the code that searches for Albo Bitossi}, label={lst:tutorial_print_result}]
	explore
			-- Tutorial code.
		local
			p1, p2: PERSON
		do
			-- Same code as before
			-- Search for Albo Bitossi with age <= 20
			print_result (query_with_composite_criterion)
		end
\end{lstlisting}

Where feature \lstinline{query_with_composite_criterion} looks like the following:
\begin{lstlisting}[language=OOSC2Eiffel, captionpos=b, caption={Invoking the code that searches for Albo Bitossi}, label={lst:tutorial_print_result}]
	query_with_composite_criterion: LINKED_LIST [PERSON]
		-- Query using a composite criterion.
		local
			query: PS_OBJECT_QUERY [PERSON]
		do
			create Result.make
			create query.make
			query.set_criterion (composite_search_criterion)
			executor.execute_query (query)

			across query as	query_result
			loop
				Result.extend (query_result.item)
			end
		end
\end{lstlisting}

As you may have noticed, it is very simple to set criteria on a query.

\section{Deletion queries}
\label{section:deletion_query}

As mentioned previously, there is another way to perform a deletion in the repository from within
\lstinline!PS_CRUD_EXECUTOR!. By calling \lstinline{execute_deletion_query} instead of \lstinline{delete}, ABEL will delete all objects in the database that would have been retrieved by executing the query normally.
You can look at the following example and compare it with its variation in Section ~\ref{section:simple_delete}.

\begin{lstlisting}[language=OOSC2Eiffel, captionpos=b, caption={Using a deletion query.}, label={lst:deletion_query}]
	delete_person (name:STRING)
		-- Delete `name' using a deletion query.
		local
			deletion_query: PS_OBJECT_QUERY [PERSON]
			criterion:PS_PREDEFINED_CRITERION
		do
			create deletion_query.make
			create criterion.make ("last_name", "=", name)
			deletion_query.set_criterion (criterion)
			executor.execute_deletion_query (deletion_query)
		end
\end{lstlisting}

It depends upon the situation if you want to use deletion queries or a direct delete command. 
Usually, a direct command is better if you already have the object in memory, whereas deletion queries are nice to use if the object is not yet loaded from the database.

\section{Tuple queries}

Consider a scenario in which you just want to have a list of all last names of persons in the database. Loading every object of type \lstinline!PERSON! might lead to a very bad performance, especially if there is a big object graph attached to each person object.

To solve this problem, \lstinline{PS_CRUD_EXECUTOR} allows to query data while returning tuples as a result. You can do this by calling feature\\

\lstinline{execute_tuple_query (a_tuple_query)}, \\

where \lstinline{a_tuple_query} is of type \lstinline{PS_TUPLE_QUERY [G]}.
The result is an iteration cursor over a list of tuples in which the attributes of an object are stored. The order of these attributes is the one defined in feature \lstinline{projection} in \lstinline{PS_TUPLE_QUERY [G]}.

\begin{lstlisting}[language=OOSC2Eiffel, captionpos=b, caption={Using tuple queries.}, label={lst:tuple_query_simple}]
	print_all_last_names
		-- Print the last name of all PERSON objects.
		local
			query: PS_TUPLE_QUERY [PERSON]
			last_name_index: INTEGER
			single_result: TUPLE
		do
			create query.make
			---- Find out at which position in the tuple the last_name is.
			last_name_index := query.attribute_index ("last_name")
			from
				executor.execute_tuple_query (query)
			until
				query.result_cursor.after
			loop
				single_result:= query.result_cursor.item
				print (single_result [last_name_index] )
			end			
		end
\end{lstlisting}

\subsection{Tuple queries and projections}
By default, a \lstinline!TUPLE_QUERY! object will only return values of attributes which are of a basic type, so no references are followed during a retrieve.
You can change this default by calling \lstinline{set_projection}, which expects an array of names of the attributes you would like to have.
If you include an attribute name whose type is not a basic one, ABEL will actually retrieve and build the attribute object, and not just another tuple.

\subsection{Tuple queries and criteria}
You are restricted to use predefined criteria in tuple queries, because agent criteria expect an object and not a tuple. You can still combine them with logical operators, and even include a predefined criterion on an attribute that is not present in the projection list. These attributes will be loaded internally to check if the object satisfies the criterion, and then they are discarded for the actual result.

\begin{lstlisting}[language=OOSC2Eiffel, captionpos=b, caption={Using tuple queries with criteria.}, label={lst:tuple_projection_selection}]
	print_last_names_of_20_year_old
		-- Print the last name of all PERSON objects with age = 20.
		local
			query: PS_TUPLE_QUERY [PERSON]
		do
			create query.make
			-- Only return the last_name of persons
			query.set_projection (<<"last_name">>)
			-- Only return persons with age = 20
			query.set_criterion (factory [["age", "=", 20]])
			from
				executor.execute_tuple_query (query)
			until
				query.result_cursor.after
			loop
				-- As we only have the last_name in the tuple,
				-- its index has to be 1.
				print (query.result_cursor.item [1] )
			end			
		end
\end{lstlisting}

\chapter{Dealing with references}
\label {chapter:references}

In ABEL, a basic type is an object of type \lstinline!STRING!, \lstinline!BOOLEAN!, \lstinline!CHARACTER! or any numeric class like \lstinline!REAL! or \lstinline!INTEGER!.
The \lstinline!PERSON! class only has attributes of a basic type. However, an object can contain references to other objects. ABEL is able to handle these references by storing and reconstructing the whole object graph (an object graph is roughly defined as all the objects that can be reached by recursively following all references, starting at some root object).

Let's look at the new class \lstinline!CHILD!:

\begin{lstlisting}[language=OOSC2Eiffel, captionpos=b, caption={The CHILD class.}, label={lst:child_class}]

class
	CHILD

create
	make

feature {NONE} -- Initialization

	make (first, last: STRING)
			-- Create a new child.
		require
			first_exists: not first.is_empty
			last_exists: not last.is_empty
		do
			first_name := first
			last_name := last
			age := 0
		ensure
			first_name_set: first_name = first
			last_name_set: last_name = last
			default_age: age = 0
		end

feature -- Access

	celebrate_birthday
			-- Increase age by 1.
		do
			age := age + 1
		ensure
			age_incremented_by_one: age = old age + 1
		end

feature -- Status report

	first_name: STRING
			-- The child's first name.

	last_name: STRING
			-- The child's last name.

	age: INTEGER
			-- The child's age.

feature -- Parents

	mother: detachable CHILD
			-- The child's mother.

	father: detachable CHILD
			-- The child's father.

	set_mother (a_mother: CHILD)
			-- Set a mother for the child.
		do
			mother := a_mother
		ensure
			mother_set: mother = a_mother
		end

	set_father (a_father: CHILD)
			-- Set a father for the child.
		do
			father := a_father
		ensure
			father_set: father = a_father
		end

invariant
	age_non_negative: age >= 0
	first_name_exists: not first_name.is_empty
	last_name_exists: not last_name.is_empty
end

\end{lstlisting}


This adds in some complexity: instead of having a single object, ABEL has to insert a \lstinline!CHILD!'s mother and father as well, and it has to repeat this procedure if their parent attribute is also attached. The good news are that the examples above will work exactly the same.

However, there are some additional caveats to take into consideration. 
Let's consider a simple example with \lstinline!CHILD! objects ``Baby Doe'', ``John Doe'' and ``Grandpa Doe''.
From the name of the object instances you can already guess what the object graph looks like: 

	\begin{center}
\begin{tikzpicture}[->,>=stealth',shorten >=1pt,auto,node distance=5cm,
  thick,main node/.style={rectangle,fill=white,draw}]

  \node[main node] (1) {$Baby Doe$};
  \node[main node] (2) [right of=1] {$John Doe$};
  \node[main node] (3) [right of=2] {$Grandpa Doe$};

 

  \path
    (1) edge node {$father$} (2)
    (2) edge node {$father$} (3);
\end{tikzpicture}
	\end{center}

Now if you insert ``Baby Doe'', ABEL will by default follow all references and insert every single object along the object graph, which means that ``John Doe'' and ``Grandpa Doe'' will be inserted as well.
This is usually the desired behavior, as objects are stored completely that way, but it also has some side effects we need to be aware of:

\begin{itemize}
\item Assume an insert of ``Baby Doe'' has happened to an empty database. 
If you now query the database for \lstinline!CHILD! objects, it will return exactly the same object graph as above, but the query result will actually have three items, as the object graph consists of three single \lstinline!CHILD! objects.
	
\item After you've inserted ``Baby Doe'', it has no effect if you insert ``John Doe'' or ``Grandpa Doe'' afterwards, because they have already been inserted by the first statement.
\end{itemize}

\section{Updates}

ABEL does not follow references during an update by default, so for example the following statement has no effect on the database:

\begin{lstlisting}[language=OOSC2Eiffel, captionpos=b, caption={References are not followed by default during updates.}, label={lst:reference_update}]
	celebrate_fathers_birthday (a_child: CHILD)
		-- Increase age of `a_child's father.
		require
			child_persistent: executor.is_persistent (a_child)
		do
			a_child.father.celebrate_birthday

			-- This won't have any effect
			executor.update (a_child)

			-- however, it works that way
			executor.update (a_child.father)
		end
\end{lstlisting}

Section~\ref{sec:going_deeper_in_object_graph} will tell you how do change the default settings.
 
\section{Going deeper in the Object Graph}
\label{sec:going_deeper_in_object_graph}
ABEL has no limits regarding the depth of an object graph, and it will detect and handle reference cycles correctly. 
You are welcome to test ABEL's capability with very complex objects, however please keep in mind that this may impact performance significantly.

To overcome this problem, you can either use simple object structures, or you can tell ABEL to only load or store an object up to a certain depth.
The default ABEL's behavior with respect to the object graph can be changed by using feature \lstinline{default_object_graph} in class \lstinline{PS_REPOSITORY} and passing an appropriate object of type \lstinline{PS_DEFAULT_OBJECT_GRAPH_SETTINGS}.

\chapter{Advanced Initialization}
\label{chapter:advanced_initialization}

The in-memory repository we've used so far doesn't store data permanently.
This is acceptable for testing or for a tutorial, but not in a real application.
Therefore, ABEL ships with support for a MySQL database and an SQLite database.

To use them, you have to assemble the needed parts. Let's focus on MySQL: you will need to create a \lstinline!PS_MYSQL_DATABASE! and a \lstinline!PS_MYSQL_STRINGS! object.
Then you will use them to create a \lstinline!PS_GENERIC_LAYOUT_SQL_BACKEND!, which you will need in turn to create the \lstinline!PS_RELATIONAL_REPOSITORY!.

The following little factory class shows the process for both a MySQL and an SQLite database:

\begin{lstlisting}[language=OOSC2Eiffel, captionpos=b, caption={Setting up a MySQL and a SQLite repository}, label={lst:advanced_initialization}]

class 
	REPOSITORY_FACTORY

feature -- Connection details
	
	username:STRING = "tutorial"
	password:STRING = "tutorial"

	db_name:STRING = "tutorial"
	db_host:STRING = "127.0.0.1"
	db_port:INTEGER = 3306

	sqlite_filename: STRING = "tutorial.db"

feature -- Factory methods

	create_mysql_repository: PS_RELATIONAL_REPOSITORY
		-- Create a MySQL repository
		local
			database: PS_MYSQL_DATABASE
			mysql_strings: PS_MYSQL_STRINGS
			backend: PS_GENERIC_LAYOUT_SQL_BACKEND
		do
			create database.make (username, password, db_name, db_host, db_port)
			create mysql_strings
			create backend.make (database, mysql_strings)
			create Result.make (backend)
		end

	create_sqlite_repository: PS_RELATIONAL_REPOSITORY
		-- Create an SQLite repository
		local
			database: PS_SQLITE_DATABASE
			sqlite_strings: PS_SQLITE_STRINGS
			backend: PS_GENERIC_LAYOUT_SQL_BACKEND
		do
			create database.make (sqlite_filename)
			create sqlite_strings
			create backend.make (database, sqlite_strings)
			create Result.make (backend)
		end
end

\end{lstlisting}

All examples from this tutorial work exactly the same, no matter if you use the \lstinline!PS_IN_MEMORY_REPOSITORY! or any of the provided database repositories.

\chapter{Transaction handling}

Every CRUD operation in ABEL is by default executed within a transaction. 
Transactions are created and committed implicitly, which has the advantage that - especially when dealing with complex object graphs - an object doesn't get inserted halfway in case of an error.

As a user, you also have the possibility to use transactions explicitly. 
This is done by manually creating an object of type \lstinline!PS_TRANSACTION! and using the \lstinline!*_within_transaction! features in \lstinline!PS_CRUD_EXECUTOR! instead of the normal ones.
For your convenience there is a factory method \lstinline{new_transaction} in class \lstinline!PS_CRUD_EXECUTOR!.

Let's consider an example where you want to update the age of every person by one:

\begin{lstlisting}[language=OOSC2Eiffel, captionpos=b, caption={}, label={lst:update_all_ages}]
	update_ages
		-- Increase everyone's age by one.
		local
			query: PS_OBJECT_QUERY [PERSON]
			transaction: PS_TRANSACTION
		do
			create query.make
			transaction := executor.new_transaction

			executor.execute_query_within_transaction (query, transaction)

			across query as query_result
			loop
				query_result.item.celebrate_birthday
				executor.update_within_transaction 
					(query_result.item, transaction)
			end

			transaction.commit

			-- The commit may have failed
			if transaction.has_error then
				if attached transaction.error.message as msg then
					print ("Commit has failed. Error: " + msg)
				end
			end
		end
\end{lstlisting}

You can see here that a commit can fail in some situations, e.g. when a write conflict happened in the database.
The errors are reported in the \lstinline!PS_TRANSACTION.has_error! attribute.
In case of an error, all changes of the transaction are rolled back automatically.

You can also abort a transaction manually by calling feature \lstinline{rollback} in class \lstinline{PS_TRANSACTION}.

\section{Transaction isolation levels}

ABEL supports the four standard transaction isolation levels found in almost every database system:
\begin{itemize}
 \item Read Uncommitted
 \item Read Committed
 \item Repeatable Read
 \item Serializable
\end{itemize}
The different levels are defined in \lstinline!TRANSACTION_ISOLATION_LEVEL!.
You can change the transaction isolation level by calling feature\\
 \lstinline{set_transaction_isolation_level} in class \lstinline{PS_REPOSITORY}.
The default transaction isolation level of ABEL is defined by the actual storage backend.

Please note that not every backend supports all isolation levels.
Therefore a backend can also use a more restrictive isolation level than you actually instruct it to use, but it is not allowed to use a less restrictive isolation level.

\chapter{Error handling}

As ABEL is dealing with IO and databases, runtime errors may happen. 
The library will in general raise an exception in case of an error and expose the error to the library user as an \lstinline!PS_ERROR! object.
ABEL recognizes two different kinds of errors:

\begin{itemize}

\item Irrecoverable errors:  fatal errors happening in scenarios like a dropped connection or a database integrity constraint violation.
The default behavior is to rollback the current transaction and raise an exception. 
If you catch the exception in a rescue clause and manage to solve the problem, you can continue using ABEL.

\item Recoverable errors: exceptional situations typically not visible to the user, because no exception is raised when they occur.
An example is a conflict between two transactions.
ABEL will detect the issue and, in case of implicit transaction management, retry.
If you use explicit transaction management, ABEL will just doom the current transaction to fail at commit time.
\end{itemize}

ABEL maps database specific error messages to its own representation for errors, which is a hierarchy of classes rooted at \lstinline!PS_ERROR!.
The following list shows all error classes that are currently defined.

If not explicitly stated otherwise, the errors in this lists belong to the first category (fatal errors).

\begin{itemize}
\item \lstinline!CONNECTION_PROBLEM!: A broken internet link, or a deleted serialization file.
\item \lstinline!TRANSACTION_CONFLICT!: A write conflict between two transactions. This is a recoverable error.
\item \lstinline!UNRESOLVABLE_TRANSACTION_CONFLICT!: A write conflict between implicit transactions that doesn't resolve after a retry.
\item \lstinline!ACCESS_RIGHT_VIOLATION!: Insufficient privileges in database, or no write permission to serialization file.
\item \lstinline!VERSION_MISMATCH!: The stored version of an object isn't compatible any more to the current type.
\item \lstinline!INTERNAL_ERROR!: Any error happening inside the library, e.g. a wrong SQL compilation.
\item \lstinline!GENERAL_ERROR!: Anything that doesn't fit into one of the categories above.
\end{itemize}

If you want to handle an error, you have to add a \lstinline{rescue} clause somewhere in your code.

You can get the actual error from the feature \lstinline!PS_CRUD_EXECUTOR.error! or \lstinline!PS_TRANSACTION.error! or - due to the fact that the \lstinline!PS_ERROR! class inherits from \lstinline!DEVELOPER_EXCEPTION! -
by performing an object test on Eiffel's \lstinline!EXCEPTION_MANAGER.last_exception!.

For your convenience, there is a visitor pattern for all ABEL error types. 
You can just implement the appropriate functions and use it for your error handling code.

The following code shows an example. 
Note that only some important features are shown:

\begin{lstlisting}[language=OOSC2Eiffel, captionpos=b, caption={Sample error handling using a visitor.}, label={lst:error_visitor_example}]

class
	MY_PRIVATE_VISITOR
inherit
	PS_ERROR_VISITOR

feature
	shall_retry: BOOLEAN
		-- Should my client retry the operation?

	visit_access_right_violation (
		error: PS_ACCESS_RIGHT_VIOLATION)
		-- Visit an access right violation error.
		do
			add_some_privileges
			shall_retry := True
		end

	visit_connection_problem (error: PS_CONNECTION_PROBLEM)
		-- Visit a connection problem error.
		do
			notify_user_of_abort
			shall_retry:=False
		end

end

class
	TUTORIAL

feature

	my_visitor: MY_PRIVATE_VISITOR
		-- A user-defined visitor to react to an error.

	executor: PS_CRUD_EXECUTOR
		-- The CRUD executor used throughout the tutorial.


	do_something_with_error_handling
		-- Perform some operations. Deal with errors in case of a problem.
		do
			-- Some complicated operations
		rescue
			my_visitor.visit (executor.error)
			if my_visitor.shall_retry then
				retry
			else
				-- The exception propagates upwards, and maybe
				-- another feature can handle it
			end
		end
end

\end{lstlisting}


%\chapter{Technical documentation}
%\section{Architecture overview}

The ABEL library can be splitted into a Frontend and a Backend part.
The frontend provides the main API, which is completely agnostic of the actual storage engine, 
whereas the backend provides a framework and some implementations to adapt ABEL to a specific storage engine.
The boundary between Backend and Frontend can be drawn straight through the deferred class REPOSITORY.

\subsection{Frontend}

If you've read the previos part of the documentation, then you should be quite familiar now with the frontend.
The main classes are:
 \begin{itemize}
  \item CRUD\_EXECUTOR: Provides features for CRUD operations, and does some error handling for transaction conflicts.
  \item QUERY: Collects information like the Criteria, Projection, and (through its generic parameter) the type of the object to be retrieved. It doesn't do anything by itself.
  \item TRANSACTION: Represents an ABEL transaction, and is responsible to internally propagate errors. It provides the features commit and rollback, but internally it relies on REPOSITORY to execute them.
  \item CRITERION: Its descendants provide a filtering function for retrieved objects, and it has builtin functions to generate a tree of criteria using the overloaded logical operators.
 \end{itemize}

You can see that the main objective of the frontend is to provide an easy to use, backend-agnostic API, and to collect information that the backend can consume.

\todo{Insert class diagram}


\subsection{Backend}

The frontend needs a repository which is specific to a data storage engine, and the backend provides a framework to implement these repositories (in cluster Framework).
There are also some already predefined repositories inside the backend (cluster Backends), like the IN_MEMORY_REPOSITORY.

Inside the framework, there are basically three layers of abstraction. The first is the deferred class REPOSITORY, which is still very object-oriented:
It takes Eiffel objects that may have a big object graph attached, and returns such fully loaded objects. 
It also provides some fine-grained control to restrict depth in an object graph.

The second important level of abstraction is the deferred class BACKEND_STRATEGY, which deals only with one object at once. 
The layer in between REPOSITORY and BACKEND_STRATEGY take care of disassembling the object graph into single objects, wrapped in a more explicit notion of the object graph. 
You can read in the object-relational mapping section ~\ref{section:ORM} how this is exactly done.
Anyway, the BACKEND_STRATEGY only knows the concept of a single object and of a collection, and it is responsible to map these to the actual persistence mechanism, e.g. a database with a specific layout. 

The third layer of abstraction is a set of wrappers to a database. It only wraps the connection details, and has functions to execute SQL and retrieve the result in a standardized way.
It is very useful if you want to easily swap e.g. from a MySQL database to SQLite, but it's abstraction is not perfect - i.e. it doesn't take into account the different SQL variations.



\includegraphics[width = 13cm] {class_diagram.png}

\section{Object-relational Mapping}
\label{section:ORM}

\subsection{Object graph settings}


\section{Backend abstraction}

\subsection{REPOSITORY}

\subsection{BACKEND\_STRATEGY}

\subsection{Database wrapper}


\section{Extensions}

\section{Adaption to a specific database layout}


\section{Limitations of the library}
TODO

Current limitations:
\begin{itemize}
\item Not all features available if adapted to a custom DB layout 
\item Inheritance currently not properly supported (due to limitation of INTERNAL)
\item Reals have rounding error
\item No ordering support (at least not yet)
\item
\end{itemize}

%What exactly has to be here? Is a missing feature (like ordering support) a limitation as well?


TODO

%\section{ PART 2: Technical Documentation}

%\subsection{Moved to technical documentation}
%about criteria
%It will internally build a tree of criteria, with specialized AND, OR or NOT instances as nodes and the usual predefined or agent instances as leafs.


%\subsubsection{Performance remarks}

%ABEL will try to let the backend do as much of the filtering as is possible, to reduce the overhead of e.g. network communication or building unnecessary objects.
%This especially means that ABEL will compile predefined queries to SQL if you have a relational database as a backend. 
%However, if there is an agent criterion OR-ed to a predefined criterion, the test can not be made in the database because you might get false negatives.
%Therefore, to have optimal performance, you should consider the following points:

%\begin{itemize}
%\item Try not to use agent criteria if you have a relational database backend.
%\item Try not to use OR on agent criteria.
%\item Try to keep OR-ed agent criteria as deep down the tree as possible (as the above OR-node defaults to true and thus is not checked for in the backend)
%\end{itemize}


%\chapter{Conclusions}
%\chapter{Conclusions}
In the following we will point out areas of future work for selective capture and replay for Eiffel and draw the conclusions of this thesis.


\section{Future Work}
A main target of future work on selective capture and replay for Eiffel must be to increase its performance. As pointed out before, it is necessary to change parts of the technique in order to reach an efficiency that can be compared with SCARPE.

Another area for future work are the implementation's current limitations. There are some limitations that restrict its usage to well prepared examples, for example the missing instrumentation of attribute accesses and the missing support for manifest strings. Making the current available features more robust is another area of work, for example the object ID support can be further improved so that it works under all circumstances. This requires additional example applications and test cases.

Native reflection support is crucial for the further success of selective capture and replay for Eiffel. The use of Erl-G is more to be seen as a workaround than a solution, because (a) there are some missing features, like access to selectively exported features and (b) it raises the necessary compile time of a project enabled for selective capture and replay to hours, making it necessary to have a dedicated machine to compile the projects. No developer will accept this increase of compile time on project he is working on, therefore selective capture and replay for Eiffel is doomed in productive environments, as long as there is no native reflection support in Eiffel. But native reflection support is something that this project can not influence, this feature must be provided by the language maintainers.


\section{Conclusion}
Our implementation of selective capture and replay for Eiffel shows that selective capture and replay is applicable to Eiffel. We found no fundamental problems, that could limit its implementation in Eiffel, neither for the implemented features nor the missing features.

The performance measurements show that optimizing the current implementation alone will not result in an efficiency that is comparable to SCARPE. It is necessary to change some basic techniques of the approach, for example incorporating the presented single-sided instrumentation (\sectref{lbl:single-sided_instrumentation}).

\begin{flushleft}
 
{{{
\bibliographystyle {plain}
\bibliography {./references}
}}}
\end{flushleft}
\end{document}